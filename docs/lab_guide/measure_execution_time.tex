\hypertarget{measure-the-execution-time-of-your-application}{%
\section{Measure the execution time of your
application}\label{measure-the-execution-time-of-your-application}}

You can use the time-specific function \emph{\textbf{XTime\_GetTime
(XTime *xtime)}} (available in the standalone BSP) to evaluate the
execution time of your application. This function provides direct access
to the\\
64-bit Global Counter in the PMU. This global timer (GT) is a 64-bit
incrementing counter with an auto-incrementing feature accessible to
both Cortex-A9 processors. The global timer is consistently clocked at
1/2 of the CPU frequency (the counter increments every two processor
cycles). The pre-defined value COUNTS\_PER\_SECOND directly indicates
the GC clock frequency (number of counts per second).

\begin{quote}
\textbf{\#include \textless stdio.h\textgreater{}}

\textbf{\#include "xiltimer.h"}

\textbf{int main()}

\textbf{\{}

\textbf{XTime tStart, tEnd;}

\textbf{// initialize the application}

\textbf{XTime\_GetTime(\&tStart); // start measuring time}

\textbf{// process the application}

\textbf{XTime\_GetTime(\&tEnd); // end measuring time}

\textbf{// finalize the application}

\textbf{printf("Execution took \%llu clock cycles.\textbackslash n",\\
2*(tEnd - tStart));}

\textbf{printf("Output took \%.2f us.\textbackslash n",}

\textbf{1.0*(tEnd - tStart) * 1000000/(COUNTS\_PER\_SECOND));}

\textbf{return 0;}

\textbf{\}}

Note: when measuring the performance of (parts of) your application, be
careful not to measure the time of the printfs!
\end{quote}

