\hypertarget{zynq-project-in-vivado}{%
\section{Zynq Project in Vivado}\label{zynq-project-in-vivado}}

Start the Vivado application from the desktop shortcut or command line
without closing Vitis.

\hypertarget{create-a-new-project-targeting-the-fpga-board}{%
\subsection{Create a New Project targeting the FPGA
board}\label{create-a-new-project-targeting-the-fpga-board}}

\hypertarget{open-vivado}{%
\subsubsection{Open Vivado}\label{open-vivado}}

\begin{figure}[htbp]
  \centering
  \includegraphics[width=\linewidth,keepaspectratio]{./media/image8.png}
  \caption{Open Vivado — without closing Vitis}
\end{figure}

\hypertarget{create-a-new-project}{%
\subsubsection{\texorpdfstring{Create a New Project
}{Create a New Project }}\label{create-a-new-project}}

\begin{figure}[htbp]
  \centering
  \includegraphics[width=\linewidth,keepaspectratio]{./media/image9.png}
  \caption{Open Vivado — Create a New Project}
\end{figure}

\hypertarget{name-the-project}{%
\subsubsection{\texorpdfstring{Name the project
}{Name the project }}\label{name-the-project}}

\begin{figure}[htbp]
  \centering
  \includegraphics[width=\linewidth,keepaspectratio]{./media/image10.png}
  \caption{Open Vivado — Name the project}
\end{figure}

Note: Avoid using spaces and/or special characters in your directory
path and in the filenames.

\hypertarget{select-project-type}{%
\subsubsection{Select project type}\label{select-project-type}}

\begin{figure}[htbp]
  \centering
  \includegraphics[width=\linewidth,keepaspectratio]{./media/image11.png}
  \caption{Select project type — path and in the filenames}
\end{figure}

\hypertarget{skip-add-sources-and-constraints}{%
\subsubsection{Skip Add Sources and
Constraints}\label{skip-add-sources-and-constraints}}

Click Next twice to skip Add Source and Add Constraints.

\hypertarget{board-selection}{%
\subsubsection{Board Selection}\label{board-selection}}

In the Default Part form, select Boards and then choose the board you
are using: Zybo, Zybo Z7-10 or Pynq-Z2 (check the version of your
board).

Note: if the board is not in the board list, go to Tools/Vivado Store to
install it.

Click Next, check the Project Summary and click Finish to create an
empty Vivado project.

\hypertarget{use-ip-integrator-to-create-a-new-block-design-with-the-zynq-processing-system}{%
\subsection{Use IP Integrator to create a new Block Design, with the
ZYNQ processing
system}\label{use-ip-integrator-to-create-a-new-block-design-with-the-zynq-processing-system}}

\hypertarget{create-a-block-design}{%
\subsubsection{Create a Block Design}\label{create-a-block-design}}

In the Flow Navigator, click Create Block Design (under IP Integrator)
with the design name \textbf{design\_1}.

\begin{figure}[htbp]
  \centering
  \includegraphics[width=\linewidth,keepaspectratio]{./media/image12.png}\includegraphics[width=\linewidth,keepaspectratio]{./media/image13.png}
  \caption{Create a Block Design — with the design name design\_1}
\end{figure}

\hypertarget{add-the-processing-system}{%
\subsubsection{Add the Processing
System}\label{add-the-processing-system}}

IP components can be added from the catalog by clicking the Add IP icon
(\textbf{+}) in the block diagram, or by right-clicking anywhere in the
Diagram workspace and selecting Add IP. Select the Zynq7 Processing
System and add it to the design.

\begin{figure}[htbp]
  \centering
  \includegraphics[width=\linewidth,keepaspectratio]{./media/image14.png}
  \includegraphics[width=\linewidth,keepaspectratio]{./media/image15.png}
  \caption{Create a Block Design — System and add it to the design}
\end{figure}

Click on Run Block Automation with the default settings by pressing OK.

\begin{figure}[htbp]
  \centering
  \includegraphics[width=\linewidth,keepaspectratio]{./media/image16.png}
  \caption{Create a Block Design — Click on Run Block Automation with the default settings by pressing OK}
\end{figure}

Ports are automatically added for the DDR and Fixed IO. An imported
configuration for the Zynq PS system related to the Zybo board has been
applied, which can be customized.

\begin{figure}[htbp]
  \centering
  \includegraphics[width=\linewidth,keepaspectratio]{./media/image17.png}
  \caption{Create a Block Design — applied, which can be customized}
\end{figure}

\hypertarget{configure-the-processing-block-with-only-one-uart-peripheral}{%
\subsubsection{Configure the processing block with only one UART
peripheral}\label{configure-the-processing-block-with-only-one-uart-peripheral}}

Double-click on the PS block to open its customization window.

Open the MIO configuration form and ensure \textbf{all the I/O are
deselected except UART 1}, i.e., ENET 0, USB 0, SD 0 (I/O Peripherals),
GPIO MIO (GPIO), Quad SPI Flash (Memory Interfaces), and Timer 0
(Application Processor Unit) must be deselected.

\hypertarget{ps-pl-configuration}{%
\subsubsection{PS-PL Configuration}\label{ps-pl-configuration}}

On the PS-PL Configuration window, keep the M AXI GP0 interface (AXI Non
Secure Enablement → GP Master AXI interface) and the FCLK\_RESET0\_N
option (General → Enable Clock Resets) selected.

\begin{figure}[htbp]
  \centering
  \includegraphics[width=\linewidth,keepaspectratio]{./media/image20.png}
  \caption{PS-PL Configuration — option (General → Enable Clock Resets) selected}
\end{figure}

Figure : PS-PL Configuration.

\hypertarget{clock-configuration}{%
\subsubsection{Clock Configuration}\label{clock-configuration}}

On the Clock Configuration window PL Fabric Clocks, keep the clock
configuration with the FCLK\_CLK0 clock enabled, and select a requested
frequency of 100 MHz.

\textbf{Do not change the Input Frequency}, which is 50 MHz on the Pynq
board and 33.(3) MHz on the Zybo-Z7-10 boards.

\hypertarget{finalize-the-ps-configuration}{%
\subsubsection{Finalize the PS
Configuration}\label{finalize-the-ps-configuration}}

Apply the customization and note that the Zynq PS block includes the
GP0, clock, and reset ports:

\begin{figure}[htbp]
  \centering
  \includegraphics[width=\linewidth,keepaspectratio]{./media/image21.png}
  \caption{Clock Configuration — GP0, clock, and reset ports:}
\end{figure}

\hypertarget{specify-ip-repository}{%
\subsection{Specify IP Repository}\label{specify-ip-repository}}

Add your IP repository folder to the IP repository list in the
\emph{\textbf{Project Settings}}.

\begin{figure}[htbp]
  \centering
  \includegraphics[width=\linewidth,keepaspectratio]{./media/image22.png}
  \caption{Specify IP Repository — Add your IP repository folder to the IP repository list in the}
\end{figure}

\hypertarget{add-your-ip-to-the-zynq-design}{%
\subsection{Add your IP to the Zynq
design}\label{add-your-ip-to-the-zynq-design}}

\hypertarget{add-new-axi-lite-ip}{%
\subsubsection{Add New AXI-Lite IP}\label{add-new-axi-lite-ip}}

Click the \emph{\textbf{Add IP icon}}, search for your new IP and add it
to the design.

\begin{figure}[htbp]
  \centering
  \includegraphics[width=\linewidth,keepaspectratio]{./media/image23.png}
  \caption{Add New AXI-Lite IP — to the design}
\end{figure}

\hypertarget{connect-the-blocks}{%
\subsubsection{Connect the blocks}\label{connect-the-blocks}}

Complete the design by automatically connecting the PS and PL blocks:
run \emph{\textbf{Connection Automation}} (and select
/axil\_macc\_0/s\_axi\_BUS1) to automatically make the required
connections:

\begin{figure}[htbp]
  \centering
  \includegraphics[width=\linewidth,keepaspectratio]{./media/image24.png}
  \caption{Connect the blocks — connections:}
\end{figure}

Note that two additional blocks, Processor System Reset and AXI
Interconnect, have been automatically added to the design.

\begin{figure}[htbp]
  \centering
  \includegraphics[width=\linewidth,keepaspectratio]{./media/image25.png}
  \caption{Connect the blocks — Interconnect, have been automatically added to the design}
\end{figure}

\includegraphics[width=0.25197in,height=0.25197in]{./media/image26.png}Regenerate
the layout .

Validate the design
\includegraphics[width=0.17708in,height=0.18264in]{./media/image27.png}.

\emph{Note: if a warning message appears on
``PCW\_UIPARAM\_DDR\_DQS\_TO\_CLK\_DELAY\_0 has negative value'', you
may ignore it.}

Check the Address Editor tab and check which address region has been
assigned to the axil\_macc\_0 s\_axi\_BUS\_A Registers.\\
Note: \emph{\textbf{Do not change}} the mapping addresses generated by
Vivado.

\begin{figure}[htbp]
  \centering
  \includegraphics[width=\linewidth,keepaspectratio]{./media/image28.png}
  \caption{Connect the blocks — Note: the mapping addresses generated by}
\end{figure}

\hypertarget{system-hardware-generation}{%
\subsection{System Hardware
Generation}\label{system-hardware-generation}}

\hypertarget{generate-block-design}{%
\subsubsection{\texorpdfstring{Generate Block Design
}{Generate Block Design }}\label{generate-block-design}}

\includegraphics[width=2.16142in,height=2.81102in]{./media/image29.png}Click
on Generate Block Design (on the Flow Navigator) to generate the
Implementation, Simulation and Synthesis files for the design.

\hypertarget{create-hdl-wrapper}{%
\subsubsection{Create HDL Wrapper}\label{create-hdl-wrapper}}

\hypertarget{in-the-sources-panel-right-click-on-design_1.bd-and-select-create-hdl-wrapper-to-generate-the-top-level-vhdl-model.-leave-the-let-vivado-manager-wrapper-and-auto-update-option-selected.}{%
\subsubsection{\texorpdfstring{In the Sources panel, right-click on
\emph{design\_1.bd} and select Create HDL Wrapper to generate the
top-level VHDL model. Leave the ``\emph{Let Vivado manager wrapper and
auto-update}'' option
selected.}{In the Sources panel, right-click on design\_1.bd and select Create HDL Wrapper to generate the top-level VHDL model. Leave the ``Let Vivado manager wrapper and auto-update'' option selected.}}\label{in-the-sources-panel-right-click-on-design_1.bd-and-select-create-hdl-wrapper-to-generate-the-top-level-vhdl-model.-leave-the-let-vivado-manager-wrapper-and-auto-update-option-selected.}}

\begin{figure}[htbp]
  \centering
  \includegraphics[width=\linewidth,keepaspectratio]{./media/image31.png}
  \caption{Create HDL Wrapper — right-click design\_1.bd and select Create HDL Wrapper}
\end{figure}

The \emph{design\_1\_wrapper.vhd} file is created, added to the project,
and set as the top module in the design. (Double-click on the file name
to see the content in the Auxiliary pane.)

\begin{figure}[htbp]
  \centering
  \includegraphics[width=\linewidth,keepaspectratio]{./media/image32.png}
  \caption{Create HDL Wrapper — to see the content in the Auxiliary pane.)}
\end{figure}

\hypertarget{design-implementation}{%
\subsubsection{Design Implementation}\label{design-implementation}}

Synthesize the design (\emph{\textbf{Run Synthesis}}), implement it
(\emph{\textbf{Run Implementation}}), and generate the bitstream
(\emph{\textbf{Generate Bitstream}}).

The hardware system has been generated and can now be exported to the
Vitis IDE to develop the embedded software and verify the application on
the development board.

After implementation, you may check the Vivado reports, namely the
\emph{Utilization Report} and the \emph{Timing Summary Report}.

\begin{figure}[htbp]
  \centering
  \includegraphics[width=\linewidth,keepaspectratio]{./media/image33.png}
  \caption{Design Implementation — Utilization Report and the Timing Summary Report}
\end{figure}

\begin{figure}[htbp]
  \centering
  \includegraphics[width=\linewidth,keepaspectratio]{./media/image34.png}
  \caption{Design Implementation — Utilization Report and the Timing Summary Report}
\end{figure}

\hypertarget{export-hardware-to-the-vitis-software-platform}{%
\subsubsection{Export Hardware to the Vitis Software
Platform}\label{export-hardware-to-the-vitis-software-platform}}

In Vivado, click \emph{\textbf{File → Export → Export Hardware}}.

This design has hardware in the Programmable Logic (PL), therefore, you
must include the bitstream to be generated and included. Make sure that
the export path corresponds to your project folder.

\begin{figure}[htbp]
  \centering
  \includegraphics[width=\linewidth,keepaspectratio]{./media/image35.png}
  \caption{Design Implementation — the export path corresponds to your project folder}
\end{figure}

\begin{figure}[htbp]
  \centering
  \includegraphics[width=\linewidth,keepaspectratio]{./media/image36.png}
  \caption{Design Implementation — the export path corresponds to your project folder}
\end{figure}

