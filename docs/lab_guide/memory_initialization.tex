\hypertarget{memory-initialization-from-a-binary-file}{%
\section{Memory Initialization from a Binary
File}\label{memory-initialization-from-a-binary-file}}

You can debug your applications by initializing part of your system's
memory using data from a binary file in your computer. This basic
procedure will be used in your future projects to demonstrate your
designs. Edit the file ps7\_init.tcl as in the figure, where images.bin
is the binary file, the first number is the memory address, and the last
number is the file size.

\begin{figure}[htbp]
  \centering
  \includegraphics[width=\linewidth,keepaspectratio]{./media/image41.png}
  \caption{Memory Initialization — editing ps7\_init.tcl with binary file, memory address and file size}
\end{figure}

You must select the memory address zone so that it corresponds to a
valid data memory zone in your system and is free to store your data.
Choose a free memory zone in your system to store your data. This zone
must be adequate for the required storage and not interfere with the
memory zones used for other storage, namely for the program code and
data. \textbf{Remember that you do the memory management.}

In the figure example, the base address selected, 0x10000000, defines a
memory area to be used in the external memory (DDR).

Then and for example, you can set the base address for your vector's
storage in your C program, and use it to set constant pointers to
address the vectors:

\begin{quote}
\textbf{\#define} VEC1\_START\_ADD 0x10000000

\textbf{\#define} VEC2\_START\_ADD (VEC1\_START\_ADD+4*VEC\_SIZE)

\textbf{int} *v1 = (\textbf{int} *)(VEC1\_START\_ADD);

\textbf{int} *v2 = (\textbf{int} *)(VEC2\_START\_ADD);
\end{quote}

