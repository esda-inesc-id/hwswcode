\hypertarget{verify-the-design-in-hardware}{%
\section{Verify the Design in
Hardware}\label{verify-the-design-in-hardware}}

\hypertarget{connect-the-board-and-establish-serial-communication-from-vitiss-terminal}{%
\subsection{\texorpdfstring{Connect the board and establish
serial communication from Vitis's Terminal
}{Connect the board and establish serial communication from Vitis's Terminal }}\label{connect-the-board-and-establish-serial-communication-from-vitiss-terminal}}

Connect and power up the FPGA board. You will use one cable to connect
to the PROG/UART port of the board.

\emph{\textbf{Warning: Do not press hard when connecting the cable, as
the board connector is fragile.}}

Open a terminal with

\emph{\textbf{Vitis → Serial Monitor Terminal}}.

Select the serial port and baud rate. After clicking
\emph{\textbf{Run}}, the application will execute, and you will see the
\emph{stdout} output in the Terminal console created before.

Note that this \emph{dotprod\_0} application is software-only (only the
processor system is used).

\hypertarget{debug-your-software-using-the-application-debugger}{%
\subsection{\texorpdfstring{Debug your software using the
Application
Debugger}{Debug your software using the Application Debugger}}\label{debug-your-software-using-the-application-debugger}}

To execute the program in debugging mode, select \emph{Debug instead of
Run.} The application will start and stop in the program's first
instruction, and the IDE automatically changes to Debug view (you can
change between Design and Debug view using the top-right buttons).

Breakpoints can be set/unset by double-clicking on the blue bar on the
left of the program window.\\
The top taskbar's resume and step control buttons can control the
program flow.\\
The program variables and memory zones can be monitored on the
right-side windows.

