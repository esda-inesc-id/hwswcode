\hypertarget{course-project}{%
\section{Course Project}\label{course-project}}

\begin{itemize}
\item
  (CNN) Convolutional Neural Network classification of images.\\
  \url{http://neuralnetworksanddeeplearning.com/chap6.html}~\\
  \url{https://cs.stanford.edu/~acoates/stl10/}
\end{itemize}

\begin{quote}
1\textsuperscript{st} part: 3×3 image convolution of images from the
STL-10 images data set (88×88 RGB images).\\
The elements of the convolution kernel will be 8-bit integers.

2\textsuperscript{nd} part: classification of images from the STL-10
images data set\\
A baseline trained network will be provided (at the Support Material web
page).\\
The network weights provided will be single-precision floating-point.
\end{quote}

\hypertarget{data-inputoutput}{%
\subsection{Data input/output}\label{data-inputoutput}}

The projects will be demonstrated by initializing the external memory
with the appropriate data input and calculating the results.

The design performance will be evaluated by measuring the total
processing time of the algorithm, starting from the first read of input
data from the external memory and ending when all the resulting data is
written back to the external memory.

\hypertarget{project-schedule}{%
\subsection{Project schedule}\label{project-schedule}}

The project will be implemented in two phases.

Lab 1: HW / SW co-processing architecture \textasciitilde{} 2 weeks
(35\%)

\begin{quote}
Embedded uniprocessor with (simpler) hardware accelerator (using GP).\\
Must demonstrate application functionality (simpler processing) and
verification competencies (HLS IP co-simulation and, optionally,
Integrated Logic Analysis), and evaluate performance.
\end{quote}

\begin{enumerate}
\def\labelenumi{\roman{enumi}.}
\item
  \begin{quote}
  Introductory application: implement and evaluate integer convolution
  2D using the application and testbenches provided in the classes.
  \end{quote}
\item
  \begin{quote}
  Base application: Compute the 2D convolution for input images with
  8-bit integer pixels.
  \end{quote}

  \begin{itemize}
  \item
    \begin{quote}
    Milestone 1: Demonstrate a working software-only application.
    \end{quote}
  \item
    \begin{quote}
    Milestone 2: demonstrate HW/SW application using the axil\_conv2D IP
    connected to the GP port, including:
    \end{quote}

    \begin{enumerate}
    \def\labelenumii{\alph{enumii}.}
    \item
      HLS C/RTL co-simulation using simple testbench with 6×6 images and
      3×3 kernel
    \item
      HLS C-simulation with the full application as a testbench, with
      the 88×88 images and 3×3 kernels extended to simulate the use of
      IP.
    \item
      HW/SW embedded application executing on the Zynq PS/PL system
      (using the IP and one ARM processor).
    \end{enumerate}
  \end{itemize}
\item
  \begin{quote}
  (Optional) Simple optimization of the HW/SW application (software
  or/and hardware component) {[}2 val{]}
  \end{quote}
\item
  \begin{quote}
  (Optional) Integrated Logic Analysis {[}1 val{]}
  \end{quote}
\end{enumerate}

Lab 2: Multiprocessor system in an FPGA \textasciitilde{} 3 weeks (50\%)

\begin{itemize}
\item
  \begin{quote}
  Embedded (heterogeneous) multiprocessor(s) with hardware
  accelerator(s).\\
  Must demonstrate functionality and show parallelization/hardware
  acceleration.
  \end{quote}
\end{itemize}

