\hypertarget{introduction}{%
\section{Introduction}\label{introduction}}

This lab guide supports the Hardware/Software Co-Design course at
Instituto Superior T\'{e}cnico. It covers three laboratory assignments
that progressively build a complete embedded HW/SW system on a
Xilinx Zynq-7000 SoC development board.

\begin{description}
  \item[Lab 0] A short tutorial on the Xilinx Vitis and Vivado design
    tools and the development board. Students become familiar with the
    HLS and FPGA design flows before tackling the main project.
  \item[Lab 1] A HW/SW co-processing architecture based on a
    uniprocessor Zynq system with a hardware accelerator designed using
    HLS. The accelerator is connected to the ARM processor via an
    AXI-Lite interface. The target application is a 2D image convolution
    kernel, representative of the convolutional layers used in neural
    networks~\cite{zynqbook}.
  \item[Lab 2] An embedded heterogeneous multiprocessor system on the
    Zynq PL fabric, with one or more hardware accelerators. Students
    must demonstrate parallelization and measure the resulting speedup.
\end{description}

The course project~(Section~\ref{course-project}) targets Convolutional
Neural Network (CNN) image classification using the STL-10 dataset.
Students implement and optimize a 2D convolution engine as an HLS-based
AXI-Lite IP, integrate it into a Zynq block design in Vivado, develop
the bare-metal software application in Vitis, and measure execution time
to evaluate their hardware acceleration.

This guide walks through each step of the design flow:
designing the AXI-Lite IP with Vitis HLS~(Section~\ref{design-an-axi-lite-ip-using-hls}),
building the Zynq block design in Vivado~(Section~\ref{zynq-project-in-vivado}),
writing and running the software application~(Section~\ref{software-application}),
verifying the design on hardware~(Section~\ref{verify-the-design-in-hardware}),
loading data from a binary file into memory~(Section~\ref{memory-initialization-from-a-binary-file}),
and measuring execution time~(Section~\ref{measure-the-execution-time-of-your-application}).

Recommended supporting material for this course is listed in the
References section~\cite{slides,ug998,ug1399,zynqbook,labfiles}.

\emph{Note: some screenshots in this guide may differ slightly from what
you see on screen due to tool updates. Please consult the course website
for the latest version of this guide and the associated project files.
Xilinx/AMD tools can occasionally be unstable; do not hesitate to ask
for help.}

