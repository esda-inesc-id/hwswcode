\hypertarget{software-application}{%
\section{Software Application}\label{software-application}}

Now return to the Vitis application you left open before.

\begin{enumerate}
\def\labelenumi{\arabic{enumi}.}
\item
  Click \emph{\textbf{Embedded Development → Create Platform Component
  to create a new platform project from the Xilinx Shell Archive (XSA)
  previously created in}} Vivado. Enter the platform component name and
  choose a suitable location. Click Next.
\end{enumerate}

\begin{figure}[htbp]
  \centering
  \includegraphics[width=\linewidth,keepaspectratio]{./media/image37.png}
  \caption{Software Application — choose a suitable location. Click Next}
\end{figure}

\begin{enumerate}
\def\labelenumi{\arabic{enumi}.}
\setcounter{enumi}{1}
\item
  Browse to the hardware specification file
  (\emph{design\_1\_wrapper.xsa) and select it. Click Next.}
\end{enumerate}

\begin{figure}[htbp]
  \centering
  \includegraphics[width=\linewidth,keepaspectratio]{./media/image38.png}
  \caption{Software Application — (design\_1\_wrapper.xsa) and select it. Click Next}
\end{figure}

\begin{enumerate}
\def\labelenumi{\arabic{enumi}.}
\setcounter{enumi}{2}
\item
  The \emph{Software Specification} fields (Operating system and
  Processor) are updated to \emph{standalone} and
  \emph{ps7\_cortexa9\_0,} respectively. Click Next, review the settings
  in the next screen and click Finish.
\item
  Build the platform project by clicking Flow/\emph{Build.}
\end{enumerate}

After the project builds, you can now start developing your software
application.

\hypertarget{create-your-c-project}{%
\subsection{Create your C Project}\label{create-your-c-project}}

In this example, you will use a previously coded C program
``\emph{dotprod\_v0.c}'' that multiplies two vectors (software-only).

\begin{enumerate}
\def\labelenumi{\arabic{enumi}.}
\item
  Create a new C project by clicking \emph{\textbf{File → New Component
  → Application Project}}. Choose a name and a suitable location for
  your software component, and click Next. Target the existing hardware
  platform you have previously created, and click Next. Leave the
  default Domain, click Next. Review and click Finish. The C project
  will be created and shown in the Vitis Project Components window.
\item
  Now you can import C source files, using the C code provided as in
  this example. To import a new C source file, select the \emph{src}
  folder in the Components view, right-click, and select
  \emph{\textbf{Import-\textgreater Files..}}. Browse and import file
  \emph{dotprod\_v0.c}.
\end{enumerate}

You can set the GCC compiler settings, including the optimization level,
by selecting Settings (below your application) and clicking the
\emph{UserConfig.cmake} item\emph{. For now, leave them as default,
namely the Optimization Level set to None (-O0).}

You can also view the Linker Script by double-clicking on
\emph{Sources/src/lscript.ld}. Here you can define where your code's
data and instruction sections will be stored. Place all the sections of
your program on the OCM and leave the heap and stack sizes as 1K and 10K
bytes, respectively.

Note: if you use the printf() function extensively, you may need to
increase the heap size (e.g., to heap size = 10kB)

\begin{enumerate}
\def\labelenumi{\arabic{enumi}.}
\setcounter{enumi}{2}
\item
  \emph{Build} the project to generate \emph{the} executable file. The
  linker combines the compiled applications, including libraries and
  drivers, and produces an executable file, in Executable Linked Format
  (ELF), that is ready to execute on your processor hardware platform.
\end{enumerate}

Opening the \emph{\textbf{.elf}} (see Components/Output), you can see
the instructions generated for your C code, including their instruction
memory addresses. At the top, you can view a list of the sizes and
starting addresses of the various sections of the program, where
\emph{\textbf{size}} indicates the size of each section and
\emph{\textbf{LMA}} is the Loadable Memory Address (start address for
each section). Note that, in this case, all sections reside in the OCM
memory space (starting with base address 0x00000000).

\begin{figure}[htbp]
  \centering
  \includegraphics[width=\linewidth,keepaspectratio]{./media/image39.png}
  \includegraphics[width=\linewidth,keepaspectratio]{./media/image40.png}
  \caption{Create your C Project — memory space (starting with base address 0x00000000)}
\end{figure}

You can check the size occupied by your program by opening the file
\emph{\textbf{dotprod0.elf.size}}. In this example it occupies about 70
kB.

text data bss dec hex filename

51111 2548 17600 71259 1165b dotprod\_0.elf

Note: your program must fit in with the memory you select to store it
in!\\
Also, you must avoid trying to store multiple programs and/or data in
the same memory zones.\\
\textbf{All the memory management is done by the designer, you!}

