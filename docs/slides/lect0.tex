\documentclass [xcolor=svgnames, t] {beamer}
\usepackage[utf8]{inputenc}
\usepackage[T1]{fontenc}
\usepackage{booktabs, comment}
\usepackage[absolute, overlay]{textpos}
\useoutertheme{infolines}
\setbeamercolor{title in head/foot}{bg=internationalorange}
\setbeamercolor{author in head/foot}{bg=dodgerblue}
\usepackage{csquotes}
\usepackage[style=verbose-ibid,backend=bibtex]{biblatex}
\bibliography{bibfile}
\usepackage{amsmath}
\usepackage[makeroom]{cancel}
\usepackage{textpos}
\usepackage{tikz}
\usepackage{listings}
\graphicspath{ {./figures/} }
\usepackage{hyperref}
\hypersetup{
    colorlinks=true,
    linkcolor=blue,
    filecolor=magenta,
    urlcolor=cyan,
}

\usetheme{Madrid}
\definecolor{myuniversity}{RGB}{0, 60, 113}
\definecolor{internationalorange}{RGB}{231, 93,  42}
 	\definecolor{dodgerblue}{RGB}{0, 119,202}
\usecolortheme[named=myuniversity]{structure}

\title[Hardware/Software Codesign]{Hw/Sw-Codesin}
\subtitle{Create CPU+IP Systems}
\institute[Técnico]{Técnico}
\titlegraphic{\includegraphics[height=2.5cm]{figs/Logo.png}}
\author[José T. de Sousa]{Jos\'e T. de Sousa}
%\institute[IObundle Lda]{IObundle Lda}
\date{\today}


\addtobeamertemplate{navigation symbols}{}{%
    \usebeamerfont{footline}%
    \usebeamercolor[fg]{footline}%
    \hspace{1em}%
    \insertframenumber/\inserttotalframenumber
}

\begin{document}

\begin{frame}
 \titlepage
\end{frame}

%%%%%%%%%%%%%%%%%%%%%%%%%%%%
\logo{\includegraphics[scale=0.2]{figs/Logo.png}~%
}
%%%%%%%%%%%%%%%%%%%%%%%%%%
\begin{frame}{What is This Course About?}
\begin{center}
\begin{itemize}
    \item Designing complete computing systems: \textbf{hardware + software}
    \item Understanding \textbf{where} to run functionality: CPU or custom hardware
    \item Joint optimization of:
    \begin{itemize}
        \item Performance
        \item Power and energy
        \item Area and cost
        \item Flexibility
    \end{itemize}
    \item Learning how modern SoCs are built
    \item Implementing real systems on a \textbf{Zynq-based FPGA board}
\end{itemize}
\end{center}
\end{frame}


\begin{frame}{Why Hardware/Software Codesign?}
\begin{center}
\begin{itemize}
    \item Modern applications are increasingly \textbf{compute-intensive}
    \begin{itemize}
        \item Signal processing
        \item Computer vision
        \item Machine learning
        \item Cryptography
    \end{itemize}
    \item Pure software solutions are often:
    \begin{itemize}
        \item Too slow
        \item Too power-hungry
    \end{itemize}
    \item Pure hardware solutions are often:
    \begin{itemize}
        \item Inflexible
        \item Expensive to modify
    \end{itemize}
    \item Codesign allows us to combine the \textbf{best of both worlds}
\end{itemize}
\end{center}
\end{frame}

\begin{frame}{Course Learning Objectives}
\begin{center}
\begin{itemize}
    \item Understand the fundamental principles of \textbf{hardware/software codesign}
    \item Learn how to analyze applications and identify performance bottlenecks
    \item Decide which parts of a system should run in:
    \begin{itemize}
        \item Software on a CPU
        \item Custom hardware accelerators
    \end{itemize}
    \item Design, implement, and integrate accelerators using:
    \begin{itemize}
        \item Vitis HLS
        \item Verilog RTL
    \end{itemize}
    \item Deploy and evaluate complete systems on a Zynq-based FPGA board
\end{itemize}
\end{center}
\end{frame}

\begin{frame}{Tools and Experimental Platform}
\begin{center}
\begin{itemize}
    \item This course is \textbf{hands-on} and tool-oriented
    \item You will work with industry-standard tools:
    \begin{itemize}
        \item \textbf{Vivado} – RTL design, synthesis, implementation
        \item \textbf{Vitis HLS} – C/C++ to hardware synthesis
        \item \textbf{Verilog} – low-level hardware design
    \end{itemize}
    \item Target platform:
    \begin{itemize}
        \item Zynq SoC (ARM CPU + FPGA fabric)
        \item Zybo development board
    \end{itemize}
    \item Software will run in a \textbf{bare-metal} environment
\end{itemize}
\end{center}
\end{frame}


\begin{frame}{What Is a Computing System?}
\begin{center}
\begin{itemize}
    \item A computing system is more than just a processor
    \item It is composed of multiple interacting components:
    \begin{itemize}
        \item Processing elements (CPUs, GPUs, accelerators)
        \item Memory hierarchy (registers, caches, RAM, external memory)
        \item Interconnects and buses
        \item Input/Output devices
        \item Software stack
    \end{itemize}
    \item Performance depends on how well these components work \textbf{together}
    \item Codesign focuses on optimizing the \textbf{whole system}
\end{itemize}
\end{center}
\end{frame}










%%%%%example slides
\begin{frame}{Outline}
\begin{center}
  \begin{columns}[onlytextwidth]
    \column{0.5\textwidth}
  \begin{itemize}
  \item Introduction
  \item Project setup
  \item Instantiate a RISC-V CPU in IOb-SoC
  \item Instantiate an IP core in your SoC
  \item Write the software to drive the IP core
  \item Simulate IOb-SoC
  \item Run IOb-SoC on an FPGA board
  \item Conclusion
  \end{itemize}
    \column{0.5\textwidth}
    \begin{figure}
      \centering
      %\includegraphics[width=0.9\textwidth]{bd.pdf}
      \caption{IOb-SoC block diagram}
      \label{fig:my_label}
    \end{figure}
  \end{columns}
\end{center}
\end{frame}


\begin{frame}{Introduction}
\begin{center}
    \begin{itemize}
      \item Building processor-based systems from scratch is challenging
      \item The IOb-SoC template eases this task by providing a base Verilog SoC equipped with
        \begin{itemize}
        \item a RISC-V CPU
        \item a memory system including boot ROM, RAM, 2-level cache system and an AXI4 interface to external memory (DDR)
        \item a UART communications module
        \item an example firmware program
        \end{itemize}
      \item Users can add IP cores and software to build their SoCs
      \item This tutorial exemplifies the addition of a Timer IP core and the use of its software driver
    \end{itemize}
\end{center}
\end{frame}

\begin{frame}{Project setup}
\begin{center}
  \begin{itemize}
  \item Use a Linux real or virtual machine (see the README file to download a VM)
    \item Install {\tt nix-shell} to deal with dependencies, especially open-source simulators such as {\tt iverilog} or {\tt verilator} 
    \item Commercial EDA tools must be installed locally or on some remote server (Vivado, Quartus, Cadence, etc) 
    \item FPGA boards must be attached to your Linux machine or to some remote server
    \item Set up {\bf ssh} access key to GitHub (\url{github.com}) (using https will ask for your password many times)
    \item Follow the instructions in the IOb-SoC repository's README file to clone the repository and install the tools
  \end{itemize}
\end{center}
\end{frame}


\end{document}
